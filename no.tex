\documentclass[journal,12pt,twocolumn]{IEEEtran}

\usepackage{setspace}
\usepackage{gensymb}

\singlespacing


\usepackage[cmex10]{amsmath}

\usepackage{amsthm}

\usepackage{mathrsfs}
\usepackage{txfonts}
\usepackage{stfloats}
\usepackage{bm}
\usepackage{cite}
\usepackage{cases}
\usepackage{subfig}

\usepackage{longtable}
\usepackage{multirow}

\usepackage{enumitem}
\usepackage{mathtools}
\usepackage{steinmetz}
\usepackage{tikz}
\usepackage{circuitikz}
\usepackage{verbatim}
\usepackage{tfrupee}
\usepackage[breaklinks=true]{hyperref}
\usepackage{graphicx}
\usepackage{tkz-euclide}
\usepackage{float}

\usetikzlibrary{calc,math}
\usepackage{listings}
\usepackage{color} %%
\usepackage{array} %%
\usepackage{longtable} %%
\usepackage{calc} %%
\usepackage{multirow} %%
\usepackage{hhline} %%
\usepackage{ifthen} %%
\usepackage{lscape}
\usepackage{multicol}
\usepackage{chngcntr}

\DeclareMathOperator*{\Res}{Res}

\renewcommand\thesection{\arabic{section}}
\renewcommand\thesubsection{\thesection.\arabic{subsection}}
\renewcommand\thesubsubsection{\thesubsection.\arabic{subsubsection}}

\renewcommand\thesectiondis{\arabic{section}}
\renewcommand\thesubsectiondis{\thesectiondis.\arabic{subsection}}
\renewcommand\thesubsubsectiondis{\thesubsectiondis.\arabic{subsubsection}}


\hyphenation{op-tical net-works semi-conduc-tor}
\def\inputGnumericTable{} %%

\lstset{
%language=C,
frame=single,
breaklines=true,
columns=fullflexible
}
\begin{document}


\newtheorem{theorem}{Theorem}[section]
\newtheorem{problem}{Problem}
\newtheorem{proposition}{Proposition}[section]
\newtheorem{lemma}{Lemma}[section]
\newtheorem{corollary}[theorem]{Corollary}
\newtheorem{example}{Example}[section]
\newtheorem{definition}[problem]{Definition}

\newcommand{\BEQA}{\begin{eqnarray}}
\newcommand{\EEQA}{\end{eqnarray}}
\newcommand{\define}{\stackrel{\triangle}{=}}
\bibliographystyle{IEEEtran}
\providecommand{\mbf}{\mathbf}
\providecommand{\pr}[1]{\ensuremath{\Pr\left(#1\right)}}
\providecommand{\qfunc}[1]{\ensuremath{Q\left(#1\right)}}
\providecommand{\sbrak}[1]{\ensuremath{{}\left[#1\right]}}
\providecommand{\lsbrak}[1]{\ensuremath{{}\left[#1\right.}}
\providecommand{\rsbrak}[1]{\ensuremath{{}\left.#1\right]}}
\providecommand{\brak}[1]{\ensuremath{\left(#1\right)}}
\providecommand{\lbrak}[1]{\ensuremath{\left(#1\right.}}
\providecommand{\rbrak}[1]{\ensuremath{\left.#1\right)}}
\providecommand{\cbrak}[1]{\ensuremath{\left\{#1\right\}}}
\providecommand{\lcbrak}[1]{\ensuremath{\left\{#1\right.}}
\providecommand{\rcbrak}[1]{\ensuremath{\left.#1\right\}}}
\theoremstyle{remark}
\newtheorem{rem}{Remark}
\newcommand{\sgn}{\mathop{\mathrm{sgn}}}
\providecommand{\abs}[1]{\left\vert#1\right\vert}
\providecommand{\res}[1]{\Res\displaylimits_{#1}}
\providecommand{\norm}[1]{\left\lVert#1\right\rVert}
%\providecommand{\norm}[1]{\lVert#1\rVert}
\providecommand{\mtx}[1]{\mathbf{#1}}
\providecommand{\mean}[1]{E\left[ #1 \right]}
\providecommand{\fourier}{\overset{\mathcal{F}}{ \rightleftharpoons}}
%\providecommand{\hilbert}{\overset{\mathcal{H}}{ \rightleftharpoons}}
\providecommand{\system}{\overset{\mathcal{H}}{ \longleftrightarrow}}
%\newcommand{\solution}[2]{\textbf{Solution:}{#1}}
\newcommand{\solution}{\noindent \textbf{Solution: }}
\newcommand{\cosec}{\,\text{cosec}\,}
\providecommand{\dec}[2]{\ensuremath{\overset{#1}{\underset{#2}{\gtrless}}}}
\newcommand{\myvec}[1]{\ensuremath{\begin{pmatrix}#1\end{pmatrix}}}
\newcommand{\mydet}[1]{\ensuremath{\begin{vmatrix}#1\end{vmatrix}}}
\numberwithin{equation}{subsection}
\makeatletter
\@addtoreset{figure}{problem}
\makeatother
\let\StandardTheFigure\thefigure
\let\vec\mathbf
\renewcommand{\thefigure}{\theproblem}
\def\putbox#1#2#3{\makebox[0in][l]{\makebox[#1][l]{}\raisebox{\baselineskip}[0in][0in]{\raisebox{#2}[0in][0in]{#3}}}}
\def\rightbox#1{\makebox[0in][r]{#1}}
\def\centbox#1{\makebox[0in]{#1}}
\def\topbox#1{\raisebox{-\baselineskip}[0in][0in]{#1}}
\def\midbox#1{\raisebox{-0.5\baselineskip}[0in][0in]{#1}}
\vspace{3cm}
\title{ASSIGNMENT 1}
\author{Valli Devi Bolla}
\maketitle
\newpage
\bigskip
\renewcommand{\thefigure}{\theenumi}
\renewcommand{\thetable}{\theenumi}
Download all python codes from
\begin{lstlisting}
https://github.com/Vallidevibolla/valli/blob/main/Collinear.py
\end{lstlisting}
%
and latex-tikz codes from
%
\begin{lstlisting}
https://github.com/Vallidevibolla/valli/blob/main/problem14.tex
\end{lstlisting}
%
\section{Question No.14}
Find the value of $k$,if the points \myvec{k\\3}, 
\myvec{6\\-2} and \myvec{-3\\4} are collinear.
\section{Solution}
Let
\begin{align}
\vec{A}=\myvec{k\\3} \\
\vec{B}=\myvec{6\\-2}\\
\vec{C}=\myvec{-3\\4}
\end{align}
 
As,given that the points are collinear,
\begin{align}
\myvec{6&-2\\-3&4\\k&3}
\end{align}
\begin{align}
\frac{1}{2}\mydet{1&1&1\\A&B&C}&=0\\
\implies\mydet{1&1&1\\k&6&-3\\3&-2&4}&=0\\
\implies -6K-9&=0\\
\implies {k}&={-3/2}
\end{align}
\section{Alternative Method}
As,given that the points are collinear,
\begin{align}
\myvec{6&-2\\-3&4\\k&3}
\end{align}
\begin{align*}\\

\myvec{k-6&3-(-2)\\6-(-3)&-2-4}\\
\rightarrow
\myvec{k-6&5\\9&-6}
\xleftrightarrow[\rowinterchange]{\myvec{R_2\rightarrow R_1}}
\myvec{9&-6\\k-6&5}\\
\xrightarrow[\rowdivision]{\myvec{R_1\rowdivision/3}}
\myvec{3&-2\\k-6&5}\\
\longrightarrow
\myvec{3&-2\\0&3\times5-(-2\times(k-6)}
\end{align*}
\implies{15+2K-12{=}0}\\ 
\implies{k}={-3/2}

 $\therefore$ Finally the value of $k$ is $\frac{-3}{2}$
 
 
 \begin{figure}[h!]
\includegraphics[width=\linewidth]{Collinear.png}
  \caption{collinear}
  \label{collinear points}
\end{figure}
%
\subsection{prove that}
These points are collinear and forms a line 
\begin{align}
\vec{A}=\myvec{\frac{-3}{2}\\3} \\
\vec{B}=\myvec{6\\-2}\\
\vec{C}=\myvec{-3\\4}
\end{align}
\begin{doc}
 The direction vectors of $AB$ and $BC$ are 
\begin{align}
\vec{B}-\vec{A} &= \myvec{\frac{15}{2}\\-5}
\\
\vec{C}-\vec{A} &= \myvec{\frac{-3}{2}\\1}
\end{align}
%
If $\vec{A}, \vec{B}, \vec{C}$ form a line, then, $AB$ and $AC$ should have the same direction vector. Hence, there exists a $k$ such that
\begin{align}
\vec{B}-\vec{A} &= k\brak{\vec{C}-\vec{B}}
\\
\implies \vec{B} &= \frac{k\vec{C} +\vec{A}}{k+1}
\label{eq:tri_geo_ex_caorth_section}
\end{align}
%
\begin{align}
 \vec{B}-\vec{A} &= k\brak{\vec{C}-\vec{A}}
\\
\implies {\myvec{\frac{15}{2}\\-5} &= k \myvec{\frac{-3}{2}\\1}}
\\
\implies {\vec{k} &= $-5$}
\end{align}
Since 
\begin{align}
\vec{B}-\vec{A} &= k\brak{\vec{C}-\vec{A}},
\end{align}
%
the points are collinear and form a line.  
Hence it is proved

 \end{doc}
 
\end{document}
